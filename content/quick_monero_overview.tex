% this file is called up by main.tex
% content in this file will be fed into the main document

% ---------------------------------------------------------------------------

\chapter{Monero Overview} 
\label{chap:overview}

In what follows we will provide a basic explanation of the Monero transaction protocol and blockhain mechanics.

Monero users have private and public keys. However, these are not directly used so as to avoid linkability to one and the same recipient. Instead, monero uses One-time keys for every single transaction and recipient.

\section{One-time keys}
\label{sec:generating-one-time-keys}

Sender and Receiver agree on a shared secret using a Diffie-Hellman exchange. The one-time public key is then calculated from the sender's factor in the secret key and the true public key of the receiver.


Once the one-time keys have been generated, the sender can mark the transaction with their corresponding public part.

Using the properties derived from the Diffie-Hellman exchange, the receiver is able to recognize transactions addressed to him. 







