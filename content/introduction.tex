% this file is called up by main.tex
% content in this file will be fed into the main document

% ---------------------------------------------------------------------------

\chapter{Introduction}
\label{chap:introduction}

The purpose of blockchains is to furnish trust to operations between unrelated parties, without requiring the collaboration of a trusted third party.

Trust is attained through the use of cryptographic artifacts which cater for virtual immutability and non-falsifiability of data registered in a readily accessible database --- the blockchain.
In other words, a blockchain is a public distributed database, containing data whose legitimacy cannot 
be disputed by any party.
\\

Cryptocurrencies store transactions in the blockchain. The latter acts as a public ledger of all the consensuated currency operations. Most cryptocurrencies store transactions in clear text, to facilitate the verification of transactions by the community.

Clearly, an open blockchain defies any basic understanding of privacy, since it virtually 
{\em publicizes} complete transaction histories of its users. 
\\

To address the lack of privacy, users of cryptocurrencies such as Bitcoin can obfuscate transactions by using temporary intermediate addresses \cite{DBLP:journals/corr/NarayananM17}. However, in spite of such measures, with appropriate tools it is possible to analyze flows and to a large extent 
link true senders with receivers \cite{DBLP:journals/corr/ShenTuY15b, DK-police-tracing-btc, Andrew-Cox-Sandia}.

In contrast, the cryptocurrency Monero, attempts to tackle the issue of privacy by storing only stealth, single-use addresses in the blockchain. In this manner, there will be no effective way of linking senders with receivers nor tracing the origins of funds \cite{Monero-intro}.

Additionally, transaction amounts in the Monero blockchain are concealed behind cryptographic constructions, so as to render more complicated to infer currency flows.

The result is a high level of privacy, possibly unmatched by other common cryptocurrencies.





\section{Objectives}
\label{sec:goals}

Monero is a cryptocurrency of recent creation, yet it displays a steady growth in popularity\footnote{
As of December 28\nth, 2017, Monero occupies the 10\nth position as regards market capitalization, see\\ \url{https://coinmarketcap.com/}}. 
Unfortunately, there is little comprehensive documentation available describing the mechanisms it uses. 
Even worse, important parts of the theoretical framework in the currency have been published
in non peer-reviewed papers which are incomplete and/or contain errors.
For significant parts of the theoretical framework of Monero, only the source code is reliable enough as source
of information.
  
We intend to palliate this situation by collecting in-depth information about Monero's inner workings, 
reviewing algorithms and cryptographic schemes, and by discussing the degree to which they might afford sufficient 
transaction privacy and security to its users.
\\ 
\\ 

We have centered our attention on release 0.11.1.0 of the Monero software suite,
the most recent release at the moment this is written.
All transaction related mechanisms described here belong to this version. 
Deprecated transaction schemes have not been explored to any extent, even if they may
be partially supported for backward compatibility reasons.


\section{Readership}

We expect the reader to possess a basic understanding of discrete mathematics and algebraic structures, but possibly only
fundamental insights in the field of cryptography.
We also expect the user to have a basic understanding of how a cryptocurrency like Bitcoin works.

A reader with this background should be able to follow our constructive, step-by-step description of the elements of the Monero cryptocurrency.

We have omitted on purpose some mathematical technicalities, when they would be in the way of clarity.
We have also omitted concrete implementation details where we thought they were not essential.
Our objective has been to present the subject half-way between mathematical cryptography and 
computer programming, aiming at completeness and conceptual clarity.

Using a consistent
notation, a succinct and single-threaded explanation, 
we believe that it is possible to lead a reader with this background to a deep understanding
of the essential aspects of the Monero cryptocurrency.  


\section{Origins of the Monero cryptocurrency}

The cryptocurrency Monero, originally known as BitMonero, was created in April, 2014, as a derivative of the proof-of-concept currency CryptoNote.

The latter is a cryptocurrency devised by an individual or team under the pseudonym of Nicolas van Saberhagen. His/their work was published in October 2013 in \cite{cryptoNoteWhitePaper}.
It offered sender and receiver anonymity through the use of one-time addresses, and untraceability of flows
by means of ring signatures.

Since its inception, Monero has further strengthened its privacy aspects by implementing amount hiding as described by
Greg Maxwell, among others in \cite{Signatures2015BorromeanRS}, as well as Shen Noether's improvements
on ring signatures \cite{ledger34}.
  

\section{Outline}

As hinted earlier, our aim has been to deliver a self-contained and step-by-step description of the Monero
cryptocurrency. This report has been structured to fulfill this objective and
lead the reader through all elements needed to describe the inner workings of the currency.
\\

In our quest for comprehensiveness, we have chosen to present all the basic elements of cryptography
needed to understand the complexities of Monero. In Chapter \ref{chap:basicConcepts} we develop essential
aspects of Elliptic Curve cryptography.

Chapter \ref{chapter:ring-signatures} outlines the ring signature related algorithms that will be applied to achieve
confidential but linkable transactions.

In Chapter \ref{chapter:pedersen-commitments} we introduce the cryptographic mechanisms used to conceal amounts.

Finally, with all the components in place we will be able to expose the transaction
schemes used in Monero in Chapter \ref{chapter:transactions}.


%While seemingly outside the main focus of this thesis, we also found that
%the consensus algorithm in Monero would interest readers. 
%We feel that it is a valuable contribution in the cryptocurrency world,
%as it brings consensus  power back to the plain participants in the network. In doing so it ameliorates security
%and resilience to attacks relying on computing power. We have dedicated Chapter  \ref{chapter:consensus}
%to putting forth the principles of the consensus algorithm in Monero.


Appendices \ref{appendix:RCTTypeFull} and \ref{appendix:RCTTypeSimple} describe the structure of sample
transactions in the blockchain, providing a connection between the theoretical elements described in earlier sections with 
their real-life implementation.







