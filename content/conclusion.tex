% this file is called up by main.tex
% content in this file will be fed into the main document

% ---------------------------------------------------------------------------

\chapter{Conclusion}
\label{chapter:conclusion}

The main goal of this thesis has been to present a complete picture of the cryptographic mechanisms used in Monero
to attain confidential transactions. There is a notable lack of documentation of the currency, and in particular,
the cryptographic aspects are only described in incomplete and non peer-reviewed papers, which contain important
errors.

All of this justified our endeavour. We have aimed at producing a self-contained, consistent and single-threaded description,
with sufficient, but not overwhelming, mathematical rigor. Balancing on a thin line between mathematical cryptography 
and computer science,
we think we have successfully targeted readers from both fields without sacrificing rigor nor applicability.


\section{Is privacy synonymous with opacity?}

Cryptocurrencies have often been associated with obscure underworld transactions.
At first sight it might seem that the privacy mechanisms offered by Monero are a step further
on the way to make transactions even more opaque to justifiable insight.

However, this is not necessarily the case.
We have shown that it is possible to meet the desire for privacy of cryptocurrency users, yet
allowing for transparency when needed. 

An authority could have access to the transaction history
of a user without putting at risk privacy, as it is commonly understood.
This is conveyed by the segregation of roles of user keys, where one allows {\em viewing}
and another one {\em spending}. 
\\

In other words, it is possible to attain the benefits of a decentralized blockchain
without compromising lawful transparency.



\section{Should you use or invest in Monero?}

In view of the hype currently surrounding cryptocurrencies, many a reader will no doubt wonder if {\em investing} in
cryptocurrencies is a good idea. We have not aimed at all at answering such questions in this thesis.

In our opinion, however, at its core, a currency is only a means of facilitating exchange of wares. Without 
financial instruments
in a currency it is hardly possible to {\em invest}. Existing cryptocurrencies do not have financial instruments
at the moment this is written, with the only feeble exception of the newly created Bitcoin Futures.
\\

In sum, we do not think it is possible to {\em invest}, properly speaking, in cryptocurrencies. 
At most, one can speculate or
bet on rising exchange rates.

The current rise in exchange rates is undoubtedly due to the ever increasing flows of money into the
currencies, and the fact that supply is limited. This contributes further to characterize so-called
investments as speculation.
\\

Monero is still a relatively small cryptocurrency. As we have shown here, the cryptographic artifacts it uses
are purposeful and well designed. However, our impression is that there is still a gap in research
around confidentiality. In Chapter \ref{chapter:privacy} we presented 2 research papers that
tried to quantify the degree of untraceability of transactions. They are in no way conclusive and we think
more is needed to be able to reasonably ascertain whether transactions are truly confidential.

For mass-scale usage, the currency has the problem of the space it consumes in the blockchain, which in the end
limits the maximum throughput in terms of transactions.

Also, we feel that the Monero Research Lab should adopt a more structured approach to software releases.
Our impression is that some releases contain important last-minute decisions. Additionally, the standard
code base suffers from readability issues, a consequence of the fact that it has been produced mainly
by a few non-software developers. For production usage of something as important as a currency,
one would desire a more disciplined software engineering approach.
\\

Should one use the Monero currency? Perhaps, but being aware of the real risks.

   
 

\section{Future work}
\label{sec:futureWork}

We believe that our work could have an intrinsic value for the Monero community. As mentioned, there is little documentation
available. In consequence, we intend to make an edited version of this thesis available to the community.

Beyond that, we are intrigued about coming developments of new signature schemes, which consume \(\mathcal{O}(\log n)\)
space, and which would allow for larger ring sizes.
As we have hinted earlier, ring sizes as well as a careful selection of members is critical to ensure untraceability.
Therefore, a new signature scheme consuming space as described would be a welcome development and worth researching.




